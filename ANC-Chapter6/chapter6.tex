\documentclass{beamer}

\usepackage{graphicx,helvet,textpos,times,verbatim,tikz,enumerate,url}%
\usepackage{wrapfig}
\usepackage{color}
\usepackage{extarrows}
\usetheme{metropolis}           % Use metropolis theme




\title{Applied Nonlinear Control \\
        \large Chapter 6 ~-~ Feedback Linearization}
\author{\large Shuai Qian}
\date{\today}
\institute{School of Automation \\
        Nanjing University of Science and Technology}


\begin{document}
  \maketitle

%contents---------------------------
  \begin{frame}
  \addtocounter{framenumber}{-2}
  \frametitle{Table of Contents}
  \thispagestyle{empty}
  \tableofcontents
  \end{frame}


  \section{1. Introduction}

  \begin{frame}{Introduction}
    \textbf{Central ideal}
    \begin{itemize}
      \item Algebraically transform a nonlinear system dynamics into a (fully or partly) linear one, so that linear control techniques can be applied.
    \end{itemize}

    \textbf{Differences between feedback linearization and conventional linearization}
    \begin{itemize}
      \item Feedback linearization : achieved by exact state transformations and feedback.
      \item Conventional linearization(Jacobian linearization) : linear approximations of the dynamics.
    \end{itemize}
   \end{frame}


%--------------------------------
   \begin{frame}{Jacobian Linearization}

    Consider a nonlinear autonomous system (\ref{nonlinear})
    \begin{equation}\label{nonlinear}
      \dot{\textbf{x}} = \textbf{f(x)}
    \end{equation}
    Assume that $\textbf{f(x)}$ is continuously differentiable,
    then the system can be written as
    $$
    \dot{\textbf{x}} = (\frac{\partial \textbf{f}}{\partial \textbf{x}})_{\textbf{x}=0} + \textbf{f}_{h.o.t.}(\textbf{x})
    $$
    (0 is an equilibrium point, and $\textbf{f}(0) = 0$).
    Let $\textbf{A}$ denotes the Jacobian matrix of $\textbf{f}$ with respect to $\textbf{x}$ at $\textbf{x}=0$
    $$
    \textbf{A} = (\frac{\partial \textbf{f}}{\partial \textbf{x}})_{\textbf{x}=0}
    $$
    Then, the system
    $$
    \dot{\textbf{x}} = \textbf{Ax}
    $$
   \end{frame}


%---------------------------------
  \section{2. Intuitive Concepts}

  \begin{frame}{Example}
    Consider the control of the level $h$ of fluid in a tank (Figure \ref{tank}) to a specified level $h_{d}$.

    %place the figure and word together!
    \begin{wrapfigure}{l}{4cm}
    \vspace{-10pt}
    \includegraphics[width=4cm]{image/tank.pdf}\\
    \vspace{-15pt}
    \caption{Fluid level control in a tank}\label{tank}
    \vspace{-10pt}
    \end{wrapfigure}

    Variables:
    \begin{itemize}
      \item Control input : $u$
      \item Cross section of the tank : $A(h)$
      \item Cross section of the outlet \\ pipe : $a$
      \item Desired level : $h_{d}$
    \end{itemize}

    The dynamic model of the tank is
    \begin{equation}\label{dynamic}
      \frac{d}{dt}\left[\int_{0}^{h}A(h)dh\right] = u(t) - a \sqrt{2gh}
    \end{equation}

  \end{frame}


  \begin{frame}{Feedback Linearization}
  The dynamic (\ref{dynamic}) can be rewritten as
  $$ A(h)\dot{h} = u-a\sqrt{2gh} $$
  If we choose $u(t)$ as
  $$ u(t) = a\sqrt{2gh} + A(h)v \quad \xlongequal{linear} \quad \dot{h}=v $$

  with $v$ being an ``equivalent input" to be specified. 
  Chooseing $v$ as
  $$ v=-\alpha \widetilde{h} \quad \xlongequal{close-loop} \quad \dot{h}+\alpha \widetilde{h}=0 $$
  with $\widetilde{h} = h(t)-h_{d}$ being the level error ($\widetilde{h}(t)\rightarrow 0 ~\text{as}~ t \rightarrow \infty$), and $\alpha$ being a strictly positive constant.
  \end{frame}
  
  
  \begin{frame}{Feedback Linearization, ctd'}
  Finally, the actual input flow is determined by the nonlinear control law
  $$ u(t) = a\sqrt{2gh} - A(h)\alpha \widetilde{h} $$
    The idea of canceling the nonlinearities and imposing a desired linear dynamics, can be simply applied to a class of nonlinear systems described by the so-called \textbf{companion form, or controllability canonical form}: 
    \begin{equation}\label{companion}
      x^{(n)} = f(\textbf{x}) + b(\textbf{x})u
    \end{equation}
    \vspace{-15pt}
    \begin{itemize}
      \item $u$ : scalar control input
      \item $x$ : scalar output of interest
      \item $\textbf{x} = \left[ x,\dot{x},\dots,x^{(n-1)}\right]^{T}$ : state vector
      \item $f(\textbf{x}), b(\textbf{x})$ : nonlinear functions of the states
    \end{itemize}
  \end{frame}
  
  
  \begin{frame}{Feedback Linearization, ctd'}
  In state-space representation, (\ref{companion}) can be written
  $$
  \frac{d}{dt}\left[\begin{array}{c}
                      x_{1} \\
                      \dots \\
                      x_{n-1} \\
                      x_{n}
                    \end{array}\right] = \left[\begin{array}{c}
                                                 x_{2} \\
                                                 \dots \\
                                                 x_{n} \\
                                                 f(\textbf{x}+b(\textbf{x})u)
                                               \end{array}\right]
  $$
  Using the control input ($b \neq 0$)\footnote{In our example, $b=\frac{1}{A(h)}, ~ f=-\frac{a\sqrt{2gh}}{A(h)}$}
  
  \begin{equation}\label{input}
    u = \frac{1}{b}\left[v-f\right]
  \end{equation}
  %(in example, $b=\frac{1}{A(h)}, ~ f=-\frac{a\sqrt{2gh}}{A(h)}$) \\
  we can cancel the nonlinearities and obtain the simple input-output relation
  $$ x^{(n)} = f(\textbf{x}) + b(\textbf{x})u \quad \xlongequal{E.q.(\ref{input})} \quad x^{(n)} = v $$
  \end{frame}
  
  
  \begin{frame}{Feedback Linearization, ctd'}
    Thus, the control law
    $$ v = -k_{0}x-k_{1}\dot{x}- \dots - k_{n-1}x^{(n-1)} $$
    with the $k_{i}$ chosen so that the polynomial $p^{n}+k_{n-1}p^{n-1} + \dots + k_{0}$ has all its roots strictly in the left-half complex plane, leads to the exponentially stable dynamics
    $$
    x^{n}+k_{n-1}x^{n-1}+\dots+k_{0}x = 0
    $$
    which implies that $x(t) \rightarrow 0$.
  \end{frame}
  
  \begin{frame}{Feedback Linearization, ctd'}
    For tasks involving the tracking of a desired output $x_{d}(t)$, the control law
    \begin{equation}\label{tracking}
      v = x_{d}^{(n)} - k_{0}e - k_{1}\dot{e}-\dots-k_{n-1}e^{(n-1)}
    \end{equation}
    where $e(t) = x(t)-x_{d}(t)$ is the tracking error, the control law leads to exponentially convergent tracking.
    \par \vspace{-5pt}
    \textcolor{red}{In our example}, $x_{d}(t)=h_{d}$ is constant, so choose 
    $$v=-\alpha(h(t) - h_{d})$$
    with the $\alpha$ leads to the exponentially stable dynamics
    $$
    \dot{x} + \alpha x=0
    $$
    that is the polynomial $p+\alpha=0$ has all its roots strictly in the left-half complex plane
    $$
    p = -\alpha \quad \xlongequal{p < 0} \quad \alpha > 0
    $$

  \end{frame}
  
  
\end{document}









