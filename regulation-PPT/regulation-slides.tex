%-------------------------------------------------------------------------
% Class options include:
% print, %colors will be adjusted for printing
% handout, %pauses will be disabled
% aspectratio=<parameter>, parameter = 1610,169,149,141,54,43(default),32
% e.g. \documentclass[aspectratio = 1610]{beamer}
%-------------------------------------------------------------------------

\documentclass{beamer}

\usetheme{ink} % theme ink

%-------------------------------------------------------------------------
% Loading packages
%-------------------------------------------------------------------------
\usetikzlibrary{shapes,arrows,decorations.pathmorphing,backgrounds,positioning,fit,petri}
\usepackage[ruled,vlined]{algorithm2e}
\usepackage{wrapfig}
\usepackage{subfig}
\usepackage{amsfonts}

\usepackage{MnSymbol,wasysym}
\usepackage{listings}
\usepackage[framed,numbered,autolinebreaks,useliterate]{mcode}

%-------------------------------------------------------------------------
% Title page
%-------------------------------------------------------------------------
\title{2-DOF Helicopter's Internal Model Design}
\subtitle{Report for Nonlinear System and Regulator Theory}
\author{Shuai Qian}
\email{shuai.qian.anc@gmail.com}
%if using linebreak "\\", then also put a "\\" at text end
\institute{School of Automation \\ Nanjing University of Science and Technology \\}
%\inframeauthor{X.Qiu}
%\inframeinstitute{Institution Name}
\date{\today}



%-------------------------------------------------------------------------
% Begin
%-------------------------------------------------------------------------
\begin{document}

\maketitle

% outline
\begin{chapterframe}
    \frametitle{Outline}
    \tableofcontents
\end{chapterframe}


\section{\large 1. Introduction}
% ===================================

\begin{frame}[fragile]
	\frametitle{{\color{red} Introduction}}
A two-degree-of-freedom (2-DOF) Helicopter is an interesting way to study simple cases of \textbf{multivariable systems.}
    \begin{figure}[h]
         \centering
         \subfloat[Picture of real products]{
         \label{fig:subfig_a}
         \includegraphics[scale=0.42]{gfx/quanser_real.jpg}
         }
         %\hspace{10pt}%
         \subfloat[Simple free-body diagram]{
         \label{fig:subfig_b}
         \includegraphics[scale=0.31]{gfx/quanser_dia.jpg}
         }
         \caption{Quanser Aero Experiment}
         \label{fig:subfig}
    \end{figure}
\end{frame}


\begin{frame}[fragile]
	\frametitle{{\color{red} Modeling for 2-DOF Helicopter}}
Given the linear state-space equations of the 2-DOF helicopter:
\begin{equation}\label{state-space}
    \left\{\begin{matrix}
    \dot{x}(t) = Ax(t)+Bu(t) \\
    y = Cx(t)+Du(t)
    \end{matrix}
    \right.
\end{equation}
where
$$
x^{T} = [\theta(t), \psi(t), \dot{\theta}(t), \dot{\psi}(t)]
$$

$$
y^{T} = [\theta(t), \psi(t)]
$$

$$
u^{T} = [V_{p}(t)~V_{y}(t)]
$$
\begin{itemize}
  \item $\theta$ and $\psi$, the pitch and yaw angles, respectively
  \item $V_{p}$ and $V_{y}$, voltages applied to the pitch and yaw rotors
\end{itemize}
\end{frame}


\begin{frame}[fragile]
	\frametitle{{\color{red} Modeling for 2-DOF Helicopter}}
And the system parameters are shown below
    \begin{align}
        A &= \left[ \begin{matrix}\nonumber
          0 & 0 & 1 & 0 \\
          0 & 0 & 0 & 1 \\
          -K_{s p} / J_{p} & 0 & -D_{p} / J_{p} & 0 \\
          0 & 0 & 0 & -D_{y}/J_{y}
        \end{matrix} \right] \\
        B &= \left[ \begin{matrix}\nonumber
          0 & 0 \\
          0 & 0 \\
          K_{p p} / J_{p} & K_{p y} / J_{p} \\
          K_{y p} / J_{p} & K_{yy} / J_{p}
        \end{matrix} \right] \\
        C &= \left[ \begin{matrix}\nonumber
          1 & 0 & 0 & 0 \\
          0 & 1 & 0 & 0
        \end{matrix} \right] \\
        D &= \left[ \begin{matrix}\nonumber
          0 & 0 \\
          0 & 0
        \end{matrix} \right]
    \end{align}
\end{frame}


\begin{frame}[fragile]
	\frametitle{{\color{red} Control Objective}}
\begin{itemize}
  \item Given an arbitrary reference signal $w(t)$, we want to control the pitch or yaw to \textbf{exponentially tracks the signal while reject disturbance}
  \item All the information we used is \textbf{regulated output}
\end{itemize}

\end{frame}



\section{\large 2. Problem Statement}
% =======================================

\begin{frame}[fragile]
	\frametitle{{\color{red}Problem Statement}}
Consider the LTI system
\begin{equation}\label{lti}
    \left\{\begin{aligned}
    \dot{w}(t) &=Sw(t) \\
    \dot{x}(t) &= Ax(t)+Bu(t)+Pw(t) \\
    e(t) &= Cx(t)+Qw(t)
    \end{aligned}
    \right.
\end{equation}
where state $x \in \mathbb{R}^{n}$, control $u \in \mathbb{R}^{m}$, and the regulated output $e \in \mathbb{R}^{p}$, and exogenous input $w \in \mathbb{R}^{r}$.

Assume
\begin{itemize}
  \item Nominal parameters $\{ A_{0},B_{0}, C_{0}, P_{0}, Q_{0} \}$
  \item $p=m$
  \item $S$ has \textbf{no eigenvalue with negative real part}
  \item The exogenous signal $w(t)$ includes \textbf{references} to be tracked and \textbf{disturbances} to be rejected
\end{itemize}
\end{frame}


\begin{frame}[fragile]
	\frametitle{{\color{red}Problem Statement}}
We want to find a \textbf{linear dynamic feedback control law}:
$$
\dot{\xi}(t) = F\xi(t)+Ge(t),\quad u(t) = H\xi(t)
$$
such that
\begin{itemize}
  \item When $w(t)\equiv 0$, the equilibrium $(x,\xi)=(0,0)$ of
  \begin{equation}\label{augnment}
    \begin{aligned}
      \dot{x} &= Ax+BH\xi \\
      \dot{\xi} &= GCx+F\xi
    \end{aligned}
  \end{equation}
  is \textbf{asymptotically stable}
  \item For \textbf{every initial condition} $(x(0),\xi(0),w(0))$, $\lim\limits_{t\rightarrow \infty} e(t) = 0$
\end{itemize}
\end{frame}



\begin{frame}[fragile]
	\frametitle{{\color{red} Internal Model}}
\begin{itemize}
  \item \textbf{\color{blue}Internal Model Principle}\\
  In plain terms, every good regulator must \textbf{incorporate a model of the outside world}

  \item \textbf{\color{blue}Internal Model Property}\\
  Assume $(G_{1}, G_{2})$ incorporates a p-copy internal model of matrix $S$. Then, for every matrix triplet $(A, B, C)$ so that
  $$\left[\begin{matrix}
      A & B \\
      G_{2}C & G_{1}
    \end{matrix}\right]$$
  is Hurwitz, and for every $(P, Q)$ of appropriate dimensions, the \textbf{regulator equation} has unique solution $(\Pi, \Gamma)$
\end{itemize}
\end{frame}


% =============================================
\section{\large 3. Internal Model Design and Simulation}

\begin{frame}[fragile]
	\frametitle{{\color{red} P-copy Internal Model Candidate}}
For the specified exosystem matrix $S \in \mathbb{R}^{r\times r}$, a matrix pair
    $$
    \left(\mathcal{G}_{1}, \mathcal{G}_{2}\right) \in \mathbb{R}^{d \times d} \times \mathbb{R}^{d \times p}
    $$
    $$
    \mathcal{G}_{1}=T\left[\begin{array}{cc}
    {K_{1}} & {K_{2}} \\
    {0} & {G_{1}}
    \end{array}\right] T^{-1}, \quad \mathcal{G}_{2}=T\left[\begin{array}{c}
    {K_{3}} \\
    {G_{2}}
    \end{array}\right]
    $$
and in which $(G_{1},G_{2})$ is of a block-diagonal form
\begin{equation}\label{g1g2}\nonumber
  \begin{aligned}
    G_{1} &= diag(a_{1},\dots,a_{p}) \\
    G_{2} &= diag(b_{1},\dots,b_{p})
  \end{aligned}
\end{equation}

We allow:
$$dim(K_{1})=dim(K_{2})=dim(K_{3}) = 0 ,\quad T = I$$
\end{frame}


\begin{frame}[fragile]
	\frametitle{{\color{red} Design P-copy Internal Model}}
\begin{block}{Minimal p-copy internal model\footnote{see Huang J. Nonlinear Output Regulation:theory and applications.Siam, 2004, pp.21}}
  We call the pair $(G_{1},G_{2})$ a minimal p-copy internal model of $S$ if the minimal polynomial of $a_{i}$ ,the characteristic polynomial of $a_{i}$, and the minimal polynomial of $S$ are the same for all $i=1,\dots,p$.
\end{block}
In my design
\begin{equation}\label{matrixS}\nonumber
  S = \left[\begin{matrix}
    0 & 1 & 0 \\
    -1 & 0 & 0 \\
    0 & 0 & 0
  \end{matrix}\right],\quad
  a_{1} = \left[\begin{matrix}
                  0 & 1 & 0 \\
                  0 & 0 & 1 \\
                  0 & -1 & 0
                \end{matrix}\right],\quad
  b_{1} = \left[\begin{matrix}
                  0 \\
                  0 \\
                  1
                \end{matrix}\right]
\end{equation}
Polynomial: $$\lambda(\lambda^{2}+1)$$
\end{frame}


\begin{frame}[fragile]
	\frametitle{{\color{red} Design P-copy Internal Model}}
According to the matrix $S$
	\begin{itemize}
	  \item Pitch tracking a sinusoidal signal
	  \item Yaw tracking a constant signal
	\end{itemize}
And we design
\begin{equation}\label{g1g2}
    G_{1} = \left[\begin{matrix}
                    a_{1} & 0_{3\times 3} \\
                    0_{3\times 3} & a_{2}
                  \end{matrix}\right],\quad
    G_{2} = \left[\begin{matrix}
                b_{1} & 0_{3\times 1} \\
                0_{3\times 1} & b_{2}
              \end{matrix}\right]
\end{equation}
where $a_{2}=a_{1}$ and $b_{2}=b_{1}$

Then, we get a p-copy internal model of composite system
\begin{equation}\label{internal-model}
  \dot{\xi} = G_{1}\xi + G_{2}e
\end{equation}
\end{frame}


\begin{frame}[fragile]
    \frametitle{{\color{red} Design P-copy Internal Model}}
With this p-copy internal model, the pair
\begin{equation}\label{control-pair}\nonumber
  \left(\left[\begin{matrix}
                A & 0_{4\times 6} \\
                G_{2}C & G_{1}
              \end{matrix}\right],~\left[\begin{matrix}
                                           B \\
                                           0_{6\times 2}
                                         \end{matrix}\right]\right)
\end{equation}
is stabilizable.

Then, we can configure poles to get the feedback gain $[K_{1}~ K_{2}]$.
Also, we design a Luenburger observer to get state:
\begin{equation}\label{observer}
  \dot{\hat{x}} = (A-LC)\hat{x}+Bu+Le
\end{equation}
where $L$ is such that $A-LC$ Hurwitz, and the control input is
$$
{\color{blue} u = -K_{1}\hat{x} - K_{2}\xi}
$$
\end{frame}


\begin{frame}[fragile]
    \frametitle{{\color{red} The Key Code}}
    \begin{lstlisting}
    % use the LQR to solve K
    Q = diag([10 3 1 10 1 10 30 10 20 27]);
    R = diag([0.1 0.1]);
    K = lqr(Ac,Bc,Q,R);
    K1 = K(1:2, 1:4);  K2 = K(1:2, 5:10);

    % design the observer gain, (A-LC) Hurwitz
    P2 = [-5+1i -5-1i -1+4i -1-4i];
    L = place(A', C', P2)';
    
    u = -K1*x_hat - K2*z;   % controler
    dy(1:4) = A*x + B*u + E*v;  % plant dynamics
    dy(5:7) = S*v;      % exosystem
    % the p-copy internal model
    dy(8:13) = G1*z + G2*e;
    % the observer dynamics
    dy(14:17) = (A-L*C)*x_hat + B*u + L*e;
    \end{lstlisting}
\end{frame}

    
\begin{frame}[fragile]
    \frametitle{{\color{red} Simulation Results for P-copy IM}}
    \begin{itemize}
      \item {\color{blue} Use m file in MATLAB(common pole configuration)}
    \end{itemize}
    \begin{figure}[h]
     \centering
     \subfloat[Regulator output]{
     \label{fig:subfig_a}
     \includegraphics[scale=0.2]{image/error.png}
     }
     %\hspace{10pt}%
     \subfloat[System state]{
     \label{fig:subfig_b}
     \includegraphics[scale=0.2]{image/state.png}
     }
     \subfloat[Control input]{
     \label{fig:subfig_c}
     \includegraphics[scale=0.2]{image/control_input.png}
     }
     \caption{Simulation results of the p-copy internal model}
     \label{fig:subfig}
    \end{figure}
\end{frame}


\begin{frame}[fragile]
    \frametitle{{\color{red} Simulation Results for P-copy IM}}
    \begin{itemize}
      \item {\color{blue} Use m file in MATLAB(LQR)}
    \end{itemize}
    \begin{figure}[h]
     \centering
     \subfloat[Regulator output]{
     \label{fig:subfig_a}
     \includegraphics[scale=0.2]{image/error_lqr.png}
     }
     %\hspace{10pt}%
     \subfloat[System state]{
     \label{fig:subfig_b}
     \includegraphics[scale=0.2]{image/state_lqr.png}
     }
     \subfloat[Control input]{
     \label{fig:subfig_c}
     \includegraphics[scale=0.2]{image/input_lqr.png}
     }
     \caption{Simulation results of the p-copy internal model}
     \label{fig:subfig}
    \end{figure}
\end{frame}


% put a gif of the quanser
\begin{frame}[fragile]
    \frametitle{{\color{red} Test on the Quanser Aero Experiment}}

\end{frame}


\begin{frame}[fragile]
    \frametitle{{\color{red} Design The Canonical Internal Model}}
    We want to design a canonical internal model which has the following form
    \begin{equation}\label{can}
      \left\{\begin{aligned}
               \dot{\eta} &= M\eta+Nu \\
               u &= u_{ss}+\bar{u}
             \end{aligned}\right.
    \end{equation}
    which can solve the linear robust output regulation problem.
\end{frame}


\begin{frame}[fragile]
    \frametitle{{\color{red} Design The Canonical Internal Model}}
    \begin{enumerate}
      \item [1.] {\color{blue}We first construct a controllable pair $(a_{1},b_{1})$ where}
    \end{enumerate}
\begin{equation}\label{controllable-pair}\nonumber
  a_{1}=\left[\begin{matrix}
                0 & 1 & 0 \\
                0 & 0 & 1 \\
                -1 & -2 & -3
              \end{matrix}\right], \quad
  b_{1}=\left[\begin{matrix}
                0 \\
                0 \\
                1
              \end{matrix}\right], \quad
\end{equation}
and let
\begin{equation}\label{M1N1}\nonumber
    M = \left[\begin{matrix}
                    a_{1} & 0_{3\times 3} \\
                    0_{3\times 3} & a_{2}
                  \end{matrix}\right],\quad
    N = \left[\begin{matrix}
                b_{1} & 0_{3\times 1} \\
                0_{3\times 1} & b_{2}
              \end{matrix}\right]
\end{equation}
where $a_{2}=a_{1}$ and $b_{2}=b_{1}$.

\begin{enumerate}
      \item [2.] {\color{blue}Then we calculate the minimal realization pair $(\Phi, \Psi)$ of the following system, $(\Phi, \Psi)$ is observable.}
    \end{enumerate}
    \begin{equation}\label{exosystem}\left\{
      \begin{aligned}
        \dot{w} &= Sw \\
        u_{ss} &= \Gamma w
      \end{aligned}\right.
\end{equation}
\end{frame}


\begin{frame}[fragile]
    \frametitle{{\color{red} Design The Canonical Internal Model}}
    \begin{equation}\label{exosystem}\left\{
      \begin{aligned}
        \dot{\tau} &= \Phi \tau \\
        u_{ss} &= \Psi \tau
      \end{aligned}\right.
    \end{equation}
and assume $\eta_{ss} = \Sigma \tau$
    \begin{enumerate}
      \item [3.] {\color{blue}Next we solve $\Sigma$ from the following Sylvester equation}
    \end{enumerate}
    \begin{equation}\label{sylvester}
      \Sigma \Phi = M\Sigma + N\Psi
    \end{equation}
    and we have $u_{ss}=\Psi \Sigma^{-1}\eta$

    \begin{enumerate}
      \item [4.] {\color{blue}Then we configure poles for the origin system $(A,B)$ to obtain feedback gain $K$.}
    \end{enumerate}

    \begin{enumerate}
      \item [5.] {\color{blue}Finally, design a Luenburger observer (\ref{observer}) to get state.}
    \end{enumerate}
    and then we obtain $\bar{u}=-K \hat{x}$
\end{frame}


\begin{frame}[fragile]
    \frametitle{{\color{red} The Key Code}}
    \begin{lstlisting}
    % transform the steady state Uss
    PSI1 = [0 1 0;0 0 1;0 -1 0];  PHI1 = [1 0 0];
    PSI2 = PSI1;   PHI2 = PHI1;
    PSI = [PSI1 zeros(3,3);zeros(3,3) PSI2];
    PHI = [PHI1 zeros(1,3);zeros(1,3) PHI2];
    
    % solve the sylvester equation
    SIGMA = sylvester(-M, PSI, N*PHI);
    % the controller for compensate exosystem
    Uss = PHI*inv(SIGMA)*eta;

    % the controller for stabilization
    P1 = [-20+3i -20-3i -18+1i -18-1i];
    K = place(A, B, P1);
    Utue = -K*x_hat;
    % the composite controller
    U = Utue + Uss;
    \end{lstlisting}
\end{frame}


\begin{frame}[fragile]
    \frametitle{{\color{red} Simulation Results for Canonical Internal Model}}
    \begin{figure}[h]
     \centering
     \subfloat[Regulator output]{
     \label{fig:subfig_a}
     \includegraphics[scale=0.2]{image/can_error.png}
     }
     %\hspace{10pt}%
     \subfloat[System state]{
     \label{fig:subfig_b}
     \includegraphics[scale=0.2]{image/can_state.png}
     }
     \subfloat[Control input]{
     \label{fig:subfig_c}
     \includegraphics[scale=0.2]{image/can_input.png}
     }
     \caption{Simulation results of the canonical internal model}
     \label{fig:subfig}
    \end{figure}

\end{frame}


\begin{frame}[fragile]
	\frametitle{{\color{red} Test on the Quanser Aero Experiment}}
	
\end{frame}



% =================================================
\section{\large 4. Deal With Input Saturation}

\begin{frame}[fragile]
    \frametitle{{\color{red} Deal With Input Saturation}}
Consider the input saturation due to physical constrains, we redesign the controller with a \textbf{saturation function} \footnote{see R.De Santis, A. Isidori.On the Output Regulation for Linear Systems in the Presence of Input Saturation, TAC, 2001.}.
    \begin{figure}[h]
     \centering
     \subfloat[Regulator output]{
     \label{fig:subfig_a}
     \includegraphics[scale=0.2]{image/error_sat.png}
     }
     %\hspace{10pt}%
     \subfloat[System state]{
     \label{fig:subfig_b}
     \includegraphics[scale=0.2]{image/state_sat.png}
     }
     \subfloat[Control input]{
     \label{fig:subfig_c}
     \includegraphics[scale=0.2]{image/input_sat.png}
     }
     \caption{Simulation results of the input saturation}
     \label{fig:subfig}
    \end{figure}
\end{frame}


% ================================================
\section{\large 5. Conclusion}

\begin{frame}[fragile]
	\frametitle{{\color{red} Conclusion}}
    \begin{itemize}
      \item Design a linear dynamic feedback controller based on the internal model principle
      \item The controller designed with the use of error feedback was able to solve the output regulation problem
      \item Use the LQR method and saturation function to solve the problem of input constraint
    \end{itemize}
\end{frame}



\begin{thanksframe}
    \frametitle{Thank you!}
\end{thanksframe}


\end{document}
